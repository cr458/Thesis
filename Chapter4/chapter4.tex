%*******************************************************************************
%****************************** Fourth Chapter *********************************
%*******************************************************************************

\chapter{Defects in III-Nitride Microdisk Cavities}

\ifpdf
    \graphicspath{{Chapter2/Figs/Raster/}{Chapter2/Figs/PDF/}{Chapter2/Figs/}}
\else
    \graphicspath{{Chapter2/Figs/Vector/}{Chapter2/Figs/}}
\fi


\section[Short title]{Background}

Microcavities possess particular optical properties due to their ability to confine light. By matching the dimensions of the cavity to the wavelength of confined light, effects such as low-threshold lasing, enhanced nonlinear conversion and directional luminescence can be achieved. Confining a dipole within a microcavity affects its emissive properties by modifying the photon density of states. Under strong coupling, where the interaction between the dipole and cavity photons occurs on shorter timescales than the average dissipation rate \cite{Reithmaier2004}, important quantum information processing tasks can be achieved such as controlled coherent coupling and the entanglement of quantum systems \cite{Hennessy2007}. Weakly-coupled microcavity systems, where cavity dissipation dominates the dipole-cavity photon interaction, have many applications in optoelectronic devices such high efficiency, low threshold lasers \cite{Vahala2003} and embedded single photon emitters \cite{Jarjour2007}.
\\ Though various geometries for cavities exist, in this chapter we will specifically address III-nitride microdisk cavities.