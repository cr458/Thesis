\chapter{Concluding Remarks}

The work in this thesis has consisted of the use and development of various microscopy techniques in order to perform correlated measurements on III-nitride photonic devices and materials, relating their structural and optical properties.

We have utilised multiple microscopy techniques to identify the cause of inhomogeneous EL in III-nitride LEDs. Hyperspectral EL and SEM-CL mapping was performed over a set of inhomogeneities in order to determine their spectral properties. Examination of the features in the SEM-CL revealed the presence of large hexagonal defects at the centre of the inhomogeneities. A TEM lamella containing the apex of the hexagonal defect was produced using a FIB-based technique based on the deposition of marker layers. This allowed for STEM-EDX mapping of the region surrounding the defect apex, revealing the presence of Al from the AlGaN EBL in the active region. Conventional TEM techniques were used to observe a bundle of TDs associated with the hexagonal defect, establishing that the defects are likely due to sub-optimal template growth conditions. APSYS simulations demonstrated the presence of \textit{p}-doped AlGaN can indeed enhance local EL through the injection of holes into the active region, establishing the presence of \textit{p}-doped material within the active region due to the hexagonal defect as a likely cause for the inhomogeneities observed.

Factors hindering the fabrication of III-nitride cavity structures were studied using a combination of TEM and tomography. The presence of unetched material on the underside of microdisk cavities has been shown to deleteriously affect the cavity's optical properties. Using FIB sample preparation methods we have extracted a portion of unetched material from the underside of a microdisk cavity and identified the presence of an edge dislocation in this material. Furthermore we have show the feasibility in using FIB-tomography in studying the roughness and damage caused by improper PEC etching. Wehave also extended these FIB-based sample preparation and tomography methods to the study of 1-D photonic crystal cavities in the form of suspended nanobeams fabricated using a two-step etching process. Due to the complex geometric requirements for these cavities relative to microdisks we have found that both electron tomography and FIB tomography provide methods to elucidate issues in the fabrication process related to the etching and lithography, albeit with several intrinsic limitations to both the ET and FIBT experiments which require consideration when interpreting the data.

The effect of Si on the morphology and optical properties of GaN/InGaN core-shell rods was studied using a correlated microscopy approach. Panchromatic CL revealed the effect of silane concentration and exposure time during the rod growth on the optical properties of the microrods. By performing STEM-HAADF imaging on cross-sectional TEM lamellae prepared from the rods we established the presence of an $\mathrm{SiN_{x}}$ layer preceding the first QW of the active region was observed, whilst also correlating poorly emitting regions in the CL with regions containing a large amount of defects such as stacking faults and voids, as well as poor surface morphology. By quantifying the Si content of the $\mathrm{SiN_{x}}$ in different regions of the rod and relating this to the coverage of the $\mathrm{SiN_{x}}$ we established that lower $\mathrm{SiN_{x}}$ coverage results in more complete coalescence of the microrod sidewalls, thus resulting in uniform QW morphology and emission along the length of the microrod.

It is important to note the crucial role the FIB/SEM dual-beam instrument has played in this research involving correlated microscopy. The versatility and accuracy of the ion-beam milling allows for the preparation of TEM samples from specific regions of samples, albeit in a destructive manner. The destructive nature of the ion-beam and the amorphisation and gallium implantation it can induce in crystalline III-nitride samples is also an important consideration: in the preparation of thin samples the induced damage can run throughout the thickness of the specimen if not controlled. Nonetheless, we have shown throughout this work that the dual-beam instrument effectively bridges the gap between site-specific luminescence spectroscopy such as CL and nanoscopic structural and compositional analysis techniques such as STEM. With regards to the use of the dual-beam instrument to perform tomography, we note the limiting factor in terms of the depth resolution which is the slice thickness. In order to further improve the depth resolution of FIBT reconstructions, future experiments may involve the use of instruments such as the ORION nano-fab which contains additional neon and helium ion-beams and enables slicing down to thicknesses of 5 nm. This type of instrument would also provide an increase in imaging resolution (approximately 0.5 nm), thus improving the overall resolution of the FIBT experiments. We also note the general improvement this type of instrument would provide for the TEM sample preparation process, as metallic contamination is eliminated when using inert gas ion beams for micromachining processes.

The incorporation of further microscopy techniques may help in providing more depth to these correlated studies. Particularly in cases where the roughness of surfaces is studied, such as the etched microcavities and the microrods, AFM may be useful in providing an accurate benchmark against which to compare roughness values evaluated by ET and FIBT. PL spectroscopy would be particularly useful in evaluating the Q-factors for the microdisks and nanobeams studied by tomography, and in conjunction with FDTD simulations may provide an excellent platform from which to optimise the fabrication of such cavities. In the study of the effect of Si on the growth of microrods, it may be useful to apply aberration-corrected HRTEM to the $\mathrm{SiN_{x}}$ combined with DFT calculations to elucidate the stochiometry and structure of the non-polar masking layer, as has been performed by Market \textit{et al.} for the polar equivalent.    

The surprising robustness of III-nitride photonic devices to defects, when compared to other III-V semiconductor materials has  perhaps ironically made the study of defects a crucial part of research in III-nitride materials. This is a theme present throughout this work: we have presented the manner in which defects can distort the EL of LEDs, their disruptive nature in the PEC etching of undercut cavities and the correlation between their presence and the quenching of CL in microrods. In fact, the correlated study of the optical and structural properties of III-nitride materials may not be so fascinating without the presence of defects: it is their ability to distort their local structural and electrical properties that make this type of study relevant within this materials group.

Beyond the III-nitride group of materials, the tomography techniques studied here may be particularly useful in studying fabricated cavities in general. Q-factors of cavities are often much higher in materials outside the III-nitrides, partially due to the high chemical and thermal stability of the nitrides resulting in difficult fabrication. As such, we expect the optical properties of cavities fabricated from other material groups to be far more sensitive to variations in thickness and roughness and truly test the resolution limits of the tomography techniques presented here.

Finally, it is hoped that the work presented in this thesis has demonstrated the value in performing studies correlating the optical and structural properties of semiconductor materials, and the crucial role the FIB/SEM dual-beam instrument has played in enabling these studies. 






